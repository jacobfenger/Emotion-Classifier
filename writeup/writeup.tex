\documentclass[conference]{IEEEtran}

\usepackage{cite}
\usepackage{amsmath,amssymb,amsfonts}
\usepackage{algorithmic}
\usepackage{graphicx}
\usepackage{textcomp}
\usepackage{xcolor}

\begin{document}

\title{CSCE 689 Final Project Writeup}

\author{\IEEEauthorblockN{Jacob Fenger} 
\IEEEauthorblockA{\textit{Texas A\&M University}\\
Jacobfenger@tamu.edu}
\and
\IEEEauthorblockN{Jacob Hallenberger} 
\IEEEauthorblockA{\textit{Texas A\&M University}\\
Jhallenberger@tamu.edu}
}

\maketitle

\section{Introduction}
For this project, we trained a Convolutional Neural Network (CNN) to classify the 
emotions expressed in pictures of faces. We then created a website where users can upload 
photos of people and our model will output the predicted class. Photos that users upload 
will then be used to update our model so it gets more accurate as users upload more 
photos. 

\section{Literature Review}
Emotion recognition using facial expressions has been studied many times in the past. A 
recent paper regarding this topic is a paper titled \textit{'Emotion Recognition Using 
Facial 
Expressions'} by Tarnowski et. al. In the paper, they tried to classify seven different 
emotional states based on facial expressions. They explored two different methods for 
classification: a three nearest neighbor classifier and a two-layer neural network with 7 
neurons in the hidden layer.

They explored several different results including subject-dependent classification and 
subject-independent classification accuracies. In my opinion, the subject-independent 
results are more important as it can show a more general accuracy when it comes to 
emotion recognition. The nearest neighbor classifier had an average accuracy of 63\% while 
the neural network had an accuracy of 73\%.

One interesting tidbit mentioned in the discussion section of this paper was the 
exploration of face visibility and how it affected emotion recognition. As one would 
expect, the classification accuracy of the neural network decreased by ~20%. Different 
methods must be used to increase emotion recognition accuracy when individuals faces are 
only partially viewable.

\section{Problem Formulation} 
With the increasing popular of photo and video use everywhere in the world today, emotion 
recognition of facial expressions could be used to enhance or analyze certain aspects of 
life. Additionally, the integration of a model into an application is something that is 
not covered too much that my team wanted to learn more about.

\section{Proposed Solution}
To classify emotions of facial expressions, we used a basic Convolutional Neural Network (
CNN) as our classifier. This model is then saved for use in the application. Flask, a 
Python web-framework, is used to host the model and user upload capabilities for 
classification. Keras is used to create the CNN as well as adapt the model when new user 
data is added.

To deal with the user uploaded photos, some preprocessing is necessary before we can 
classify the image. We must grayscale the image, perform face recognition to isolate
the face, and then resize the image before classification. This is all done very
easily with OpenCV in Python.

\section{Data Description}
The data used for training and testing the classifier is from a Kaggle competition:
https://www.kaggle.com/c/
challenges-in-representation-learning-facial-expression-recognition-challenge/data

This data set consists of 35887 images of faces (48x48 pixel grayscale) that are already 
preprocessed. The classes for the faces are: angry, disgust, fear, happy, sad, surprise, 
and neutral. Figure 1 shows an example of what is contained within the dataset. 

\begin{figure}[t]
\caption{A photo from the data used.}
\includegraphics[scale=2]{face}
\centering
\end{figure}

\section{Results}
Our CNN achieved an accuracy of 64\% with the test set. Results could be improved by 
adding more data to our training set. Each emotion was also not represented evenly with 
only a few samples actually being labeled as disgust. One problem with some of the data 
that was used to train the CNN was that some photos had people's faces covered up with 
may affect certain results. One way to deal with this images is to remove them from the 
training set, or to try and acquire more data that could help with the accuracy of these 
types of images.

\section{Conclusions}
The main take away from our project is that it is not too hard to add a model to an 
existing web application. Keras and other libraries make it very easy to save models and 
also update them. Future work would be to improve classifier performance as well as make 
a better looking website for the user. Better model adaptation techniques could be looked 
at to improve performance as the user base grows.


\end{document}